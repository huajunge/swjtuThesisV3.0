%=================================================================================
%                西南交通大学研究生学位论文LaTeX模板
%=================================================================================

% 使用全局样式swjtuThesis,大部分的模板格式均在样式文件中定义完成
% 单面打印: oneside, 双面打印: twoside
\documentclass[twoside,openany]{swjtuThesis}

% 超链接颜色设置(调试时候使用,方便区分文本和超链接)
%\hypersetup{colorlinks,linkcolor=red,citecolor=green,urlcolor=magenta}
\hypersetup{colorlinks=true,pdfborder={0 0 0},citecolor=black,linkcolor=black,urlcolor=black} % 正式版

% 加载信息
%---------------------------------------------------------------------------------
%                西南交通大学研究生学位论文基本信息
%---------------------------------------------------------------------------------

% 请注意:本文件需用户填写论文的相应信息。

% 论文信息
% 定义申请学位
% 请注意:博士学位候选人输入 Doctor,硕士学位候选人输入 Master
\def\degree{Doctor} 

% 论文中文标题
% 请注意:标题控制在36个汉字以内),其中用 \\ 实现标题换行,单行不得超过18个汉字。
% 		  \underline命令用于实现标题的下划线,请把标题输入在\underline命令的的{}中
%		  V2.0, \underline{}已不再使用。
\cTitle{面向轨迹数据一体化管理的\\分布式技术研究}

% 论文英文标题(72个字符以内,包含空格)
\eTitle{swjtuThesis V3.0}


% 国内图书分类号
%\CI{CLS no.}
\CI{TP181, TP311}
% 国际图书分类号
%\UDC{UNC no.}
\UDC{004.8}
% 保密等级
\secLevel{公开}

%----------------------------
% 学生信息
% 中英文姓名
%\author{作\quad 者}
%\eAuthor{Author Name}
\author{何华均} % 盲审使用 \author{\#\#\#}
\eAuthor{Huajun He} % 盲审使用 \eAuthor{\#\#\#}

% 年级
\grade{2020}
\gradeN{2020}

% 学科(工学、理学、社会学,etc)
\cDiscipline{} % 留空
\eDiscipline{Philosophy}

% 专业(10个汉字以内)
% 请注意:对于超过10个汉字的专业,比如“防灾减灾工程及防护工程”,目前排版仍然不够美观,正在修正。
%        也请相应专业的同学提供过往师兄的硕士论文以供参考。
%\cMajor{专业名称}
%\eMajor{Major Name}
\cMajor{计算机科学与技术}
\eMajor{Computer Science and Technology}

% 导师(10个汉字以内)
% 请注意:导师姓名和导师的职称之间加上\hspace{0.1em}以增大间距,整体显得更为美观。 
%\cTutor{导师\hspace{0.1em} 教授/博导}
%\eTutor{Professor Supervisor Name}
\cTutor{郑宇、李天瑞} % \cTutor{***\hspace{0.1em} 教授}, 盲审使用-> \cTutor{***}
\eTutor{***} % \eTutor{Professor ***}, 盲审使用-> \eTutor{***}

% 封面日期,依次为年,月,日
\cDate{二零二四}{六}{一} % 大写,例如,{二零二零}{六}{一}
\eDate{2024}{June}{1} % 例如,{2020}{June}{1}				% 加载论文信息
\input{setup/type}    			% 区分硕博类型(模板自动区分,在info.tex中进行设置)
\input{setup/package}			% 加载使用宏包


\begin{document}

%---------------------------------------------------------------------------------
% 封面生成(包括中、英文封面、版权协议以及硕(博)士主要工作简述)
\pdfbookmark[0]{封面}{Cover}
\makecover

% 论文前言(包括中、英文摘要及目录,页码为罗马数字,在研究生论文目录中不显示摘要)
\frontmatter
\pagenumbering{Roman}
\thesispagestyle{-4}{西南交通大学\cDegree{}研究生学位论文}{第~\thepage~页}	% 页眉设置

% 加载中文摘要
\pdfbookmark[0]{中英文摘要}{Abstract}
\include{preface/cabstract}

% 加载英文摘要
\include{preface/eabstract}

% 目录生成
\pdfbookmark[0]{目录}{Content}
\tableofcontents

%---------------------------------------------------------------------------------
% 学位论文正文(包含各章节内容和结论,页码为阿拉伯数字)
\mainmatter
\thesispagestyle{-4}{西南交通大学\cDegree{}研究生学位论文}{第~\thepage~页}	% 页眉设置

% 加载论文正文(第1章 - 第N章、结论)
% 目前共有四个章节文件,用户可以自行在content/:文件夹中添加chap0X.tex文件
% 若新增文件为content/chap_name.tex, 则在这里直接通过\include{content/chap_name}来包含该章节内容

% V3.0更改——调整允许跨页的强度,0--4,越大则更偏向于跨页显示
\allowdisplaybreaks[4]

%---------------------------------------------------------------------------------
%                西南交通大学研究生学位论文:第一章内容
%---------------------------------------------------------------------------------

\chapter{绪论}

\section{研究背景与意义}
\section{国内外研究现状}
\section{当前研究工作中存在的挑战}
\section{本文的研究内容}
\section{本文的贡献和与创新}
\section{本文的组织结构}






\subsection{文献引用}
当用户完成在自己的.bib参考文献库文件中录入所有的参考文献信息之后,可以通过JabRef软件自动生成文献的Bibtexkey,也就是在swjtuThesis中对文献进行引用的关键词,在确保.bib库文件在main.tex主文件中被调用之后,在学位论文的任何位置只要在通过指令\verb|\cite{Bibtexkey}|即可实现对该篇文献的引用。

\par
在学位论文最后的参考文献列表中,模板会按照出现的顺序(出现多次引用的文献按第一次出现的位置为主),依照设定好的参考文献样式(国标文件:GB/T 7714-2015)列出所有文中所引用过的参考文献。

\par
本节最后,给出一段用~\BibTeX{}~实现参考文献插入的实例:

\begin{framed}
	电磁学作为物理学的一个重要分支,主要研究自然界中四种基本相互作用之一的电磁力,其基本数学描述以及物理框架早于1873年在J. C. Maxwell的经典著作\cite{Maxwell1873}中所奠定。一百多年来,经历了在数代科学家的传承和不断探索\textsuperscript{\cite{Stratton1941,Cheng1989,Jackson1999,Guru2004,Kong2008,Griffiths2012,Purcell2013,Ida2015}},电磁科学技术的研究和应用都到达了一个前所未有的高度。
	
	作为当下电磁学前沿技术应用研究之一的\textbf{非接触电能传输技术},其本质上是一种借助于空间无形软介质(如磁场、电场、激光、微波等)实现将电能由能量发射端通过非接触的形式传递至能量拾取端的全新电能供给模式\textsuperscript{\cite{黄学良2013}}。目前,在现阶段近距离的大功率非接触能量传输的研究及应用中,普遍使用磁场感应式电能传输技术(Inductive Power Transfer, IPT)\textsuperscript{\cite{covic2013inductive}}。
	
	应用在轨道交通牵引供电系统中,IPT技术与传统架空网、三轨、储能式等物理接触供电方式相比存在着十分明显的优点:无接触火花及触电危险,无积尘和接触损耗,无机械磨损,可适应多种恶劣天气环境(如下雪和积水)。综上,\textbf{IPT技术有望成为未来轨道交通牵引供电方式的重要发展方向之一},近年来,包括西南交通大学智能化牵引供电课题组在内的各国科研究团队逐渐投入对基于IPT技术的非接触牵引供电系统研究\textsuperscript{\cite{Buja2015,Kim2015}}。
\end{framed}

\par
在Version 3.0中,为了进一步兼容更多的参考文献格式(比如ArXiv),.bst文件中额外引入了新的函数,以下为新函数的示例,建议进一步参照.bst文件内的注释,.bib文件的新增示例内容以及最终生成文稿的参考文献格式使用新函数:

\begin{framed}
	这是使用\verb|@website{clevert2015ELU}|的引用效果\cite{clevert2015ELU}。

	这是使用\verb|@article{scarselli2008GNN_article}|的引用效果\cite{scarselli2008GNN_article},此时使用的是默认的\verb|@article|函数。接下来改用\verb|@freecite{scarselli2008GNN_freecite}|来实现相似的效果,注意此时已经通过\textbf{symbol}和\textbf{freeinfo}完成了对GB/T文献类型标志(\textbf{[FreeCite]})和文献相关信息(\textbf{TNNLS})的自定义\cite{scarselli2008GNN_freecite}。

	这是使用\verb|@inproceedings{vaswani2017Transformer_nopages}|的引用效果,注意此时的会议引用不包含页码信息\cite{vaswani2017Transformer_nopages},而在包含页码信息时,\verb|@inproceedings|与\verb|@book|已经能够支持无页码信息和有页码信息间引用格式的自动切换,注意此时使用\verb|@inproceedings{vaswani2017Transformer_pages}|的引用效果\cite{vaswani2017Transformer_pages}。
\end{framed}
%---------------------------------------------------------------------------------
%                西南交通大学研究生学位论文:第二章内容
%---------------------------------------------------------------------------------
\chapter{TMan系统结构概览}
%---------------------------------------------------------------------------------
%                西南交通大学研究生学位论文:第三章内容
%---------------------------------------------------------------------------------
\chapter{分布式轨迹索引和存储}
\section{轨迹和路网轨迹数据模型}
\section{轨迹索引}
\subsection{时间段编码}
\subsection{空间形状编码}
\subsection{时空编码}
\subsection{ID时间范围编码}
\section{路网轨迹索引}
\subsection{时间段编码}
\subsection{路径编码}
\section{轨迹代表性特征提取}
\subsection{基于道格拉斯算法的特征点和特征边界提取}
\subsection{路网轨迹特征提取}
\section{分布式轨迹数据存储}
\subsection{多索引表存储结构}
\subsection{辅助索引表存储结构}
\subsection{索引缓存}
\section{分布式路网轨迹数据存储}
\subsection{路网轨迹表}
\subsection{路网轨迹特征表存储}
%---------------------------------------------------------------------------------
%                西南交通大学研究生学位论文:第四章内容
%---------------------------------------------------------------------------------
\chapter{分布式轨迹查询处理算法}

% %---------------------------------------------------------------------------------
%                西南交通大学研究生学位论文:第四章内容
%---------------------------------------------------------------------------------
\chapter{轨迹时空分析应用}
\section{基于阈值的相似轨迹查询}
\section{Top-k相似轨迹查询}
\section{风险人群探查}
% %---------------------------------------------------------------------------------
%                西南交通大学研究生学位论文:第四章内容
%---------------------------------------------------------------------------------
\chapter{轨迹一体化管理引擎}
\section{引擎核心架构}
\section{数据定义语句}
\section{数据操作语句}
\section{数据查询语句}
\section{数据分析语句}
\section{应用案例}

% 加载论文结论
\include{content/conclusion}

%---------------------------------------------------------------------------------
% 论文附录(包括致谢、参考文献、证明、工作列表)

% 加载致谢文件
\include{appendix/remerciement}

% 加载参考文献
\clearpage
\phantomsection
\addcontentsline{toc}{chapter}{参考文献}
\begin{spacing}{1.2} %数字代表行距
	\setlength{\bibsep}{3pt plus1pt minus1pt}
	\bibliographystyle{ref/swjtuBST} %定义参考文献列表格式,参考国标文件GB/T 7714-2015制作
	\bibliography{ref/refEx} %加入参考文献的.bib库,可自行建立替换,refEx为示范文件
\end{spacing}

% 加载附录文件和个人工作列表
\include{appendix/appA}						% 非必要章节,主要用以放置测试数据、数学证明或者个人简介
\include{appendix/myWork}					% 攻读硕(博)士论文攻读期间取得的科研成果,论文、专利等

\end{document}